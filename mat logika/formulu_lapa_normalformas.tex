\documentclass[12pt]{article} 
\usepackage{polyglossia}
\usepackage{mathtools}
\usepackage{enumerate}
\usepackage{fancyhdr}
\usepackage{graphics}

\begin{document}

\noindent
Nesot negācijas zīmi pāri kvantoriem, kvantori mainās uz pretējiem.\\
$\neg(\forall x F(x)) \leftrightarrow \exists x\neg F(x)\\
\neg(\exists x F(x)) \leftrightarrow \forall x\neg F(x)\\
$

\noindent
1. un 2. de Morgāna likums\\
$\neg(A \lor B) \leftrightarrow (\neg A \land \neg B)\\
\neg(A \land B) \leftrightarrow (\neg A \lor \neg B)\\
$

\noindent
Dubulta negācijas likumi\\
$\neg\neg A \leftrightarrow A\\
$

\noindent
Teorēma 2.6.4\\
$(A \rightarrow B ) \leftrightarrow (\neg A \lor B )\\
$

\noindent
Distributīvie likumi\\
$(A \lor B) \land C \leftrightarrow (A \land C) \lor (B \land C) \\
(A \land B) \lor C \leftrightarrow (A \lor C) \land (B \lor C) \\
$
\end{document} 