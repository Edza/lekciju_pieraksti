\documentclass[12pt]{article} 
\usepackage{polyglossia}
\usepackage{mathtools}
\usepackage{enumerate}
\usepackage{fancyhdr}
\usepackage{graphics}

\begin{document}

\noindent
Nesot negācijas zīmi pāri kvantoriem, kvantori mainās uz pretējiem.\\
$\neg(\forall x F(x)) \leftrightarrow \exists x\neg F(x)\\
\neg(\exists x F(x)) \leftrightarrow \forall x\neg F(x)\\
$

\noindent
1. un 2. de Morgāna likums\\
$\neg(A \lor B) \leftrightarrow (\neg A \land \neg B)\\
\neg(A \land B) \leftrightarrow (\neg A \lor \neg B)\\
$

\noindent
Dubulta negācijas likumi\\
$\neg\neg A \leftrightarrow A\\
$

\noindent
Teorēma 2.6.4\\
$(A \rightarrow B ) \leftrightarrow (\neg A \lor B )\\
$

\noindent
Distributīvie likumi\\
$(A \lor B) \land C \leftrightarrow (A \land C) \lor (B \land C) \\
(A \land B) \lor C \leftrightarrow (A \lor C) \land (B \lor C) \\
$

Normālformas:
\begin{enumerate}
\item
Priekšējā normālforma (Prenex normalform) kvantori izteiksme sākumā
 $\forall x \exists y ... \forall z  ... (A(x,y)\lor B(z) \land C(x) ...)$
\item
Skolema normālforma. Izslēdzam visus eksistences kvantorus ($\exists x$), aizvietojam $x$ ar funkcijām $f(...)$, kur to argumenti ir visi mainīgie un universālkvantoriem ($\forall r$) \textbf{pa kreisi} no x.\\
Piemēram, $\forall r \exists x \forall y \exists z$, $x$ aizvieto ar $f(r)$, t.i., $x \leftarrow f(r)$, un $z \leftarrow g(r,y)$
\item
Konjunktīvā normālforma UN-i ($\land$) no VAI-iem ($\lor$) no predikātiem ($P(x,y, ...)$). Pie predikātiem var būt negācijas ($\neg P(x,y, ...)$).\\
$(A(x)\lor \neg B(y))\land (A(x)\lor \neg P(y) \lor C(z)) \land (\neg A(x)\lor B(y) \lor X(z1))$
\item
Klausulu forma. Konjuktīvās normālformas VAI-i katrs jaunā rindā - klauzulā. (Aizvietot $\land$ ar komatu un jaunu rindu)\\
$(A(x)\lor \neg B(y)),\\
 (A(x)\lor \neg P(y) \lor C(z)),\\
 (\neg A(x)\lor B(y) \lor X(z1))$
\end{enumerate}
\end{document} 